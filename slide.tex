\documentclass[dvipdfmx,12pt]{beamer}
\usepackage{bxdpx-beamer}
\usepackage{pxjahyper}
\usepackage{minijs}
\usepackage{amsmath}
\usepackage{pxrubrica} 
\renewcommand{\kanjifamilydefault}{\gtdefault}
\usetheme{AnnArbor}
\usepackage{multicol}
\usepackage{color}


\setbeamertemplate{theorems}[numbered]
\newtheorem{dfn}{定義}
\newtheorem{thm}{定理}
\newtheorem{conjecture}{予想}
\def\languagename{ja}
\deftranslation[to=ja]{Theorem}{定理}
\deftranslation[to=ja]{Definition}{定義}
\deftranslation[to=ja]{Conjecture}{予想}

\title{恒差数列}
\author{三星 連}
\date{2019年11月3日}
\begin{document}
\maketitle

\section*{もくじ}
\begin{frame}
\frametitle{もくじ}    
\tableofcontents
\end{frame}

\section{算術三角形と数列}

\subsection{算術三角形}
\begin{frame}
\frametitle{算術三角形}
\begin{table}
%\begin{center}
\begin{tabular}{cccccccccccccccc}
&&&&&&&&&&&&&&&\\
\multicolumn{7}{c}{}&\multicolumn{2}{c}{1}&\multicolumn{7}{c}{}\\
\multicolumn{6}{c}{}&\multicolumn{2}{c}{1}&\multicolumn{2}{c}{1}&\multicolumn{6}{c}{}\\
\multicolumn{5}{c}{}&\multicolumn{2}{c}{1}&\multicolumn{2}{c}{2}&\multicolumn{2}{c}{1}&\multicolumn{5}{c}{}\\
\multicolumn{4}{c}{}&\multicolumn{2}{c}{1}&\multicolumn{2}{c}{3}&\multicolumn{2}{c}{3}&\multicolumn{2}{c}{1}&\multicolumn{4}{c}{}\\
\multicolumn{3}{c}{}&\multicolumn{2}{c}{1}&\multicolumn{2}{c}{4}&\multicolumn{2}{c}{6}&\multicolumn{2}{c}{4}&\multicolumn{2}{c}{1}&\multicolumn{3}{c}{}\\
\multicolumn{2}{c}{}&\multicolumn{2}{c}{1}&\multicolumn{2}{c}{5}&\multicolumn{2}{c}{10}&\multicolumn{2}{c}{10}&\multicolumn{2}{c}{5}&\multicolumn{2}{c}{1}&\multicolumn{2}{c}{}\\
&\multicolumn{2}{c}{1}&\multicolumn{2}{c}{6}&\multicolumn{2}{c}{15}&\multicolumn{2}{c}{20}&\multicolumn{2}{c}{15}&\multicolumn{2}{c}{6}&\multicolumn{2}{c}{1}&\\
\multicolumn{2}{c}{1}&\multicolumn{2}{c}{7}&\multicolumn{2}{c}{21}&\multicolumn{2}{c}{35}&\multicolumn{2}{c}{35}&\multicolumn{2}{c}{21}&\multicolumn{2}{c}{7}&\multicolumn{2}{c}{1}
\end{tabular}
%\end{center}
\end{table}
算術三角形の各項は$_nC_r=\frac{n!}{r!(n-r)!}$で表せる。
\end{frame}

\begin{frame}
\frametitle{算術三角形と2の冪数列\footnote{自作語です}}
\begin{alignat*}{2}
&1& &=1\\
1&+1& &=2\\
1+&2+1& &=4\\
1+3&+3+1& &=8\\
1+4+&6+4+1& &=16\\
1+5+10&+10+5+1& &=32
\end{alignat*}
\end{frame}

\begin{frame}
\frametitle{算術三角形とフィボナッチ数列}
\begin{table}
\begin{tabular}{ccccccc}
1&\multicolumn{6}{c}{}\\
&1&1&\multicolumn{4}{c}{}\\
&&1&2&1&&\\
&&&1&3&3&1\\
&&&&1&4&6\\
&&&&&1&5\\
&&&&&&1 \\ \hline
1&1&2&3&5&8&13
\end{tabular}
\end{table}
\end{frame}

\subsection{二つの数列}
\begin{frame}
\frametitle{今日の目標}
\begin{itemize}
\item{
フィボナッチ数列
\begin{table}
\raggedright
\begin{tabular}{ccccccc}
1&1&2&3&5&8&13
\end{tabular}
\end{table}
}
\item{
2の冪数列
\begin{table}
\raggedright
\begin{tabular}{ccccccc}
1&2&4&8&16&32&64
\end{tabular}
\end{table}
}
\end{itemize}
今回の目標
\begin{itemize}
\item 2つの数列の関連を数式化する
\item 一般の形に広げる
\end{itemize}
\end{frame}


\subsection{前提知識}

\begin{frame}
\frametitle{フィボナッチ数列}
\begin{definition}
フィボナッチ数列$\{F_n\}$は以下の漸化式による数列である。
$$
F_1=F_2=1,\quad F_{n}=F_{n-1}+F_{n-2}
$$
\end{definition}
\begin{theorem}[ビネーの公式]
フィボナッチ数列の一般項は以下であらわされる。
$$
F_n=\frac{1}{\sqrt{5}}\left(\left(\frac{1+\sqrt{5}}{2}\right)^n-\left(\frac{1-\sqrt{5}}{2}\right)^n\right)
$$
\end{theorem}
$\frac{1+\sqrt{5}}{2}$を黄金数といい$\phi$で表す。
\end{frame}

\begin{frame}
\frametitle{2の冪数列}
\begin{definition}
2の冪数列$\{\mbox{弐}_n\}$は以下で定義される。
$$
\mbox{弐}_n=2^{n-1}
$$
\end{definition}
\end{frame}

\begin{frame}
\frametitle{二項係数}
\begin{theorem}
二項係数$_nC_r$は以下の等式を満たす。
\[_{n+1}C_{r+1}=_nC_{r+1}+_nC_r\]
\end{theorem}
\end{frame}

\begin{frame}
\frametitle{二項係数と数列}
\begin{theorem}
フィボナッチ数列は二項係数を用いて以下のようにあらわせる。
$$
F_n=\sum_{k=0}^{\lfloor \frac{n-1}{2} \rfloor}{_{n-k-1}C_k}
$$
\end{theorem}
\begin{theorem}
2の冪数列は二項係数を用いて以下のようにあらわせる。
$$
\mbox{弐}_n=\sum_{k=0}^{n-1}{_{n-1}C_k}
$$
\end{theorem}
\end{frame}

\begin{frame}
\frametitle{二項係数と数列}
\begin{multicols}{2}

\begin{alignat*}{2}
&{_0C_0}& &=1\\
{_1C_0}&+{_1C_1}& &=2\\
{_2C_0}+&{_2C_1}+{_2C_2}& &=4\\
{_3C_0}+{_3C_1}&+{_3C_2}+{_3C_3}& &=8\\
{_4C_0}+{_4C_1}+&{_4C_2}+{_4C_3}+{_4C_4}& &=16
\end{alignat*}


\begin{table}
\begin{tabular}{cccccc}
\quad&\multicolumn{5}{c}{}\\

${_0C_0}$&\multicolumn{5}{c}{}\\
&${_1C_0}$&${_1C_1}$&\multicolumn{3}{c}{}\\
&&${_2C_0}$&${_2C_1}$&${_2C_2}$&\\
&&&${_3C_0}$&${_3C_1}$&${_3C_2}$\\
&&&&${_4C_0}$&${_4C_1}$\\
&&&&&${_5C_0}$\\ \hline
1&1&2&3&5&8

\end{tabular}
\end{table}

\end{multicols}
\end{frame}

\section{恒差数列}
\subsection{恒差数列の定義}
\begin{frame}
\frametitle{階差数列}
二つの数列の階差数列を考えてみよう。
\begin{itemize}
\item{
フィボナッチ数列
$$
\Delta F_n=F_{n+1}-F_n=F_{n-1}
$$
}
\item{
2の冪数列
$$
\Delta \mbox{弐}_n=\mbox{弐}_{n+1}-\mbox{弐}_n=\mbox{弐}_n
$$
}
\end{itemize}
このように階差数列が自身となる数列を、\ruby[g]{恒}{・}に\ruby[g]{差}{・}となる数列ということで恒差数列と呼ぼう。
\end{frame}

\begin{frame}
\frametitle{恒差数列の定義}
\begin{definition}
以下の等式を満たすような非負整数kが存在するとき、数列$\{a_{n,k}\}$を恒差数列と呼ぶ。
$$
\Delta a_{n,k}=a_{n-k,k}
$$
\end{definition}
\end{frame}

\begin{frame}
\frametitle{脱線}
\begin{theorem}
黄金数の冪数列は恒差数列となる。
$$
\Delta \phi^n=\phi^{n+1}-\phi^n=\phi^{n-1}
$$
\end{theorem}
黄金数$\phi$は$\phi^2-\phi^1-\phi^0=0$を満たすから、移項して両辺に$\phi$を何度かかければいい。
\end{frame}

\subsection{恒差数列と算術三角形}
\begin{frame}
\frametitle{恒差数列の例}
初項から$k+1$項までが1となるような恒差数列を見てみよう
\begin{table}
\begin{tabular}{c||c|c|c|c|c|c|c|c|c|c}
k&1&2&3&4&5&6&7&8&9&10\\ \hline \hline
0&1&2&4&8&16&32&64&128&256&512\\ \hline
1&1&1&2&3&5&8&13&21&34&55\\ \hline
2&1&1&1&2&3&4&6&9&13&19\\ \hline
3&1&1&1&1&2&3&4&5&7&10\\ \hline
4&1&1&1&1&1&2&3&4&5&7
\end{tabular}
\end{table}
今回は、このように初項から数項が1となるものを特に考えることとする。
\end{frame}

\begin{frame}
\frametitle{算術三角形から恒差数列を作る}
\Large{
算術三角形から恒差数列を作りたい!\\
\vspace{0.5cm}
どうやって?
}
\end{frame}


\begin{frame}
\frametitle{特殊から一般化}
下の二つを見比べて一般化しよう。
\begin{multicols}{2}

\begin{alignat*}{2}
&1& &=1\\
1&+1& &=2\\
1+&2+1& &=4\\
1+3&+3+1& &=8\\
1+4+&6+4+1& &=16\\
1+5+10&+10+5+1& &=32
\end{alignat*}

\begin{table}
\begin{tabular}{ccccccc}
\quad&\multicolumn{6}{c}{}\\
1&\multicolumn{6}{c}{}\\
&1&1&\multicolumn{4}{c}{}\\
&&1&2&1&&\\
&&&1&3&3&1\\
&&&&1&4&6\\
&&&&&1&5\\
&&&&&&1 \\ \hline
1&1&2&3&5&8&13
\end{tabular}
\end{table}
\end{multicols}
\end{frame}

\begin{frame}
\frametitle{漸化式の相似}
そもそも、なぜ算術三角形の和がフィボナッチ数列などになるのかというと、二項係数の漸化式$$_{n+1}C_{r+1}=_nC_{r+1}+_nC_r$$がフィボナッチ数列の漸化式$$F_{n+2}=F_{n+1}+F_n$$や2の冪数列の漸化式$$\mbox{弐}_{n+1}
=\mbox{弐}_n+\mbox{弐}_n$$と類似しているから。
\end{frame}

\begin{frame}
\frametitle{}
実際、以下のように、二項係数の漸化式と数列の漸化式がきれいに対応している。

\begin{multicols}{2}

\begin{table}
\begin{tabular}{cccccc}
1&4&6&4&1&\\
&+&+&+&+&\\
&1&4&6&4&1\\
$\parallel$&$\parallel$&$\parallel$&$\parallel$&$\parallel$&$\parallel$\\
1&5&10&10&5&1
\end{tabular}
\end{table}
\begin{table}
\begin{tabular}{ccc}
\textcolor{red}{1}&&\\
\textcolor{green}{3}&\textcolor{green}{3}&\textcolor{red}{1}\\
\textcolor{blue}{1}&\textcolor{blue}{4}&\textcolor{green}{6}\\
&\textcolor{magenta}{1}&\textcolor{blue}{5}\\
&&\textcolor{magenta}{1}\\
\end{tabular}
\end{table}
\end{multicols}
\end{frame}

\begin{frame}
\frametitle{}
ところで、恒差数列は隣り合う項の差を考えたいので、斜めになっている数字が隣り合うように並べ替えたらうまくいくのではないかと考える。

つまり、引き算する数字を隣り合わせるということ。
\begin{multicols}{2}
\begin{table}
\begin{center}
\begin{tabular}{cccccccccccccccc}
\multicolumn{7}{c}{}&\multicolumn{2}{c}{1}&\multicolumn{7}{c}{}\\
\multicolumn{6}{c}{}&\multicolumn{2}{c}{1}&\multicolumn{2}{c}{1}&\multicolumn{6}{c}{}\\
\multicolumn{5}{c}{}&\multicolumn{2}{c}{1}&\multicolumn{2}{c}{2}&\multicolumn{2}{c}{1}&\multicolumn{5}{c}{}\\
\multicolumn{4}{c}{}&\multicolumn{2}{c}{1}&\multicolumn{2}{c}{3}&\multicolumn{2}{c}{3}&\multicolumn{2}{c}{1}&\multicolumn{4}{c}{}\\
\multicolumn{3}{c}{}&\multicolumn{2}{c}{1}&\multicolumn{2}{c}{4}&\multicolumn{2}{c}{6}&\multicolumn{2}{c}{4}&\multicolumn{2}{c}{1}&\multicolumn{3}{c}{}\\
\multicolumn{2}{c}{}&\multicolumn{2}{c}{1}&\multicolumn{2}{c}{5}&\multicolumn{2}{c}{10}&\multicolumn{2}{c}{10}&\multicolumn{2}{c}{5}&\multicolumn{2}{c}{1}&\multicolumn{2}{c}{}\\
&\multicolumn{2}{c}{1}&\multicolumn{2}{c}{6}&\multicolumn{2}{c}{15}&\multicolumn{2}{c}{20}&\multicolumn{2}{c}{15}&\multicolumn{2}{c}{6}&\multicolumn{2}{c}{1}&\\
&&&&&&&&&&&&&&&
\end{tabular}
\end{center}
\end{table}


\begin{table}[b]
\begin{center}
\begin{tabular}{cccccc}
&&&&&\\
1&1&1&1&1&1\\
1&2&3&4&5&6\\
1&3&6&10&15&21\\
1&4&10&20&35&56\\
1&5&15&35&70&126\\
1&6&21&56&126&252
\end{tabular}
\end{center}
\end{table}
\end{multicols}

\end{frame}


\begin{frame}
\frametitle{算術三角形と数列}
\begin{multicols}{2}
\begin{table}[htb]
\begin{tabular}{cccccc}
1&1&1&1&1&1\\
&1&2&3&4&5\\
&&1&3&6&10\\
&&&1&4&10\\
&&&&1&5\\
&&&&&1\\ \hline
1&2&4&8&16&32
\end{tabular}
\end{table}
\begin{table}[htb]
\begin{tabular}{cccccccc}
1&1&1&1&1&1&1&1\\
&&1&2&3&4&5&6\\
&&&&1&3&6&10\\
&&&&&&1&4\\ \hline
1&1&2&3&5&8&13&21
\end{tabular}
\end{table}
\end{multicols}

\end{frame}

\begin{frame}
\frametitle{算術三角形と数列}
\begin{multicols}{2}
\begin{table}[htb]
\begin{tabular}{cccccccc}
1&1&1&1&1&1&1&1\\
&&&1&2&3&4&5\\
&&&&&&1&3\\ \hline
1&1&1&2&3&4&6&9
\end{tabular}
\end{table}
\begin{table}[htb]
\begin{tabular}{cccccccc}
1&1&1&1&1&1&1&1\\
&&&&1&2&3&4\\ \hline
1&1&1&1&2&3&4&5
\end{tabular}
\end{table}
\end{multicols}

\end{frame}


\begin{frame}
\begin{table}[htb]
\begin{tabular}{cccccc}
${_0C_0}$&${_1C_1}$&${_2C_2}$&${_3C_3}$&${_4C_4}$&${_5C_5}$\\
&${_1C_0}$&${_2C_1}$&${_3C_2}$&${_4C_3}$&${_5C_4}$\\
&&${_2C_0}$&${_3C_1}$&${_4C_2}$&${_5C_3}$\\
&&&${_3C_0}$&${_4C_1}$&${_5C_2}$\\
&&&&${_4C_0}$&${_5C_1}$\\
&&&&&${_5C_0}$\\ \hline
1&2&4&8&16&32
\end{tabular}
\end{table}
\end{frame}

\begin{frame}
\begin{table}[htb]
\begin{tabular}{cccccccc}
${_0C_0}$&${_1C_1}$&${_2C_2}$&${_3C_3}$&${_4C_4}$&${_5C_5}$&${_6C_6}$&${_7C_7}$\\
&&${_1C_0}$&${_2C_1}$&${_3C_2}$&${_4C_3}$&${_5C_4}$&${_6C_5}$\\
&&&&${_2C_0}$&${_3C_1}$&${_4C_2}$&${_5C_3}$\\
&&&&&&${_3C_0}$&${_4C_1}$\\ \hline
1&1&2&3&5&8&13&21
\end{tabular}
\end{table}
\end{frame}

\begin{frame}
\begin{table}[htb]
\begin{tabular}{cccccccc}
${_0C_0}$&${_1C_1}$&${_2C_2}$&${_3C_3}$&${_4C_4}$&${_5C_5}$&${_6C_6}$&${_7C_7}$\\
&&&${_1C_0}$&${_2C_1}$&${_3C_2}$&${_4C_3}$&${_5C_4}$\\
&&&&&&${_2C_0}$&${_3C_1}$\\ \hline
1&1&1&2&3&4&6&9
\end{tabular}
\end{table}
\end{frame}


\begin{frame}
\frametitle{二項係数と数列}
\begin{theorem}
恒差数列は二項係数を用いて以下のようにあらわせる。
$$
a_{n,k}=\sum_{i=0}^{\lfloor \frac{n-1}{k+1}\rfloor}{_{n-ki-1}C_{n-(k+1)i-1}}=\sum_{i=0}^{\lfloor \frac{n-1}{k+1}\rfloor}{_{n-ki-1}C_{i}}
$$
\end{theorem}
\end{frame}

\begin{frame}
\frametitle{今後の課題}
\begin{itemize}
\item 一般の恒差数列を考えたい
\item 算術四面体で似たようなことができないか考えたい
\end{itemize}
\end{frame}

\begin{frame}
ご清聴ありがとうございました。
\end{frame}

\end{document}